\chapter{Strategie}
\label{chap:strategie}

\section{Produktvariation}

Smoilet 2.0 wird in 2 Varianten angeboten:

\begin{itemize}
\item End Kunden Variante
\item OEM Variante
\end{itemize}

\paragraph{End Kunden Variante}

Die End Kunden Variante muss einfach und mit wenigen Handgriffen in eine bestehnde Toilette montiert werden k�nnen. Das �usssere Erscheinungsbild der End Kunden Variante muss schlicht und �stetisch sein. Damit passt Smoilet 2.0 optisch elegant zu einer bestehenden Toilette. Smoilet 2.0 in der End Kunden Variante soll m�glichst unsichtbar an der Toilette angebracht werden k�nnen.

\paragraph{OEM Variante}

Die OEM Variante im Gegensatz zur End-Kunden Variante muss optisch keine Anforderungen erf�llen. Sie wird direkt ab Werk in ein Toiletten System eingebaut und somit optimal integriert. Der Einbau von Smoilet 2.0 wird durch eine Fachperson erledigt die gezielt geschult werden kann. Somit m�ssen Kriterien wie, einfacher Einbau mit wenigen Handgriffen, nicht erf�llt werden und somit kann die OEM Variante etwas g�nstiger produziert werden.


\section{Distributionspolitik}

Zur Distribution von Smoilet 2.0 sehen wir sowohl einen direkten Absatz, wie auch ein indirekter Absatz vor.

\subsection{Direkter Absatz}

Als Distributionskanal wird eine eigene E-Commerce Plattformen gew�hlt. Die E-Commerce Plattform ist ein einfacher simpler Web-Auftritt auf dem die End-Kunden Variante von Smoilet 2.0 gekauft werden kann. �?ber diese Plattform wird die End-Kunden Variante von smoilet 2.0 vertrieben.

\subsection{Indirekter Abastz}

Weiter wird die End-Kunden Variante von Smoilet 2.0 in Bau\&Hobby Gesch�ften f?r End-Kunden vertrieben und �ber den Online Handel auf E-Commerce Plattformen mit Produkten im Bereich Heim-Automatisierung und IOT vertrieben.

Die OEM Variante von Smoilet 2.0 wird durch einen Partner direkt in einem bestehenden Toilette-System eingebaut. Smoilet 2.0 wird somit �ber die Kan�le des Partner ausgeliefert werden.
OEM Partner k�nnen Smoilet 2.0 ebenfalls �ber unsere E-Commerce Plattform beziehen, �ber einen getrennten b2b Bereich.

\section{Marktvolumen, Marktpotential, Marktanteil}

\subsection{Markpotential}

Weltweit haben ca. 40\% Zugang zu einer Sp�ltoilette, bei einer ungef�hren Bev�lkerung von etwa 7 Milliarden Personen, ergibt das etwa 2.8 Milliarden Personen mit Zugang zu einer Sp�ltoilette. Wenn wir annehmen, dass alle Haushalte mindestens eine Toilette haben, plus alle Toiletten in Gesch�ften, Schulen, �ffentlichen Geb�uden, gibt das min. eine Toilette pro zwei Personen. Somit gibt es weltweit ca 1.4 Milliarden Toiletten.


Als Markt werden L�nder angestrebt mit einem hohen Interesse an Technology, Zu beginn konzentrieren wir uns auf die deutschsprachigen L�nder:

\begin{itemize}
\item Schweiz (ca. 8 Millionen)
\item Deuschland (ca. 80 Millionen)
\item �stereich (ca 8.5 Millionen)
\end{itemize}

Das ergibt Total ca 96 Millionnen Einwohner. Wenn wir uns das Alterssegment zwischen 20 und 39 Jahren anschauen (In der Schweiz sind das gem�ss BFS \footnote{\url{http://www.bfs.admin.ch/bfs/portal/de/index/themen/01/02/blank/key/alter/gesamt.html}} ca. 25\% der Bev�lkerung) bleiben ca 24. Millionen Einwohner �brig.
Wir denken das Smoilet 2.0 hautps�chlich von Techniker oder Personen mit gleichrangigen berufen benutzt wird. Gem�ss BFS \footnote{\url{http://www.bfs.admin.ch/bfs/portal/de/index/themen/01/07/blank/ind43.indicator.43052.430108.html}} sind das ca 20\% und somit ca 5. Millionen Einwohner.

Das ergibt mit den oben genannten Zahlen (min. eine Toilette pro zwei Personen) etwa 2.5 Millionen Toiletten.

\subsection{Marktanteil}

F�r unsere Crowdfunding Kampagne streben wir den Verkauf von 10'000 smoilet 2.0 Ger�ten an (0.4\% Marktanteil).
Anschliessend planen wir einen Wachstum von 50\% pro Jahr an.

Das ergibt folgende Marktanteile.

1. Jahr 10'000 Einheiten 0.4 Prozent
2. Jahr 15'000 Einheiten 1 Prozent
3. Jahr 22'500 Einheiten 1.9 Prozent

% Todo Make Table!




