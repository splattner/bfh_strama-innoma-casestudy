\chapter{Markt}
\label{chap:markt}

\section{Markpotential}

Weltweit haben ca. 40\% Zugang zu einer Sp�ltoilette, bei einer ungef�hren Bev�lkerung von etwa 7 Milliarden Personen, ergibt das etwa 2.8 Milliarden Personen mit Zugang zu einer Sp�ltoilette. Wenn wir annehmen, dass alle Haushalte mindestens eine Toilette haben, plus alle Toiletten in Gesch�ften, Schulen, �ffentlichen Geb�uden, gibt das min. eine Toilette pro zwei Personen. Somit gibt es weltweit ca 1.4 Milliarden Toiletten.

Als Markt werden L�nder angestrebt mit einem hohen Interesse an Technologie. Zu beginn konzentrieren wir uns auf die deutschsprachigen L�nder:

\begin{itemize}
\item Schweiz (ca. 8 Millionen)
\item Deutschland (ca. 80 Millionen)
\item �sterreich (ca 8.5 Millionen)
\end{itemize}

Das ergibt Total ca. 96 Millionen Einwohner. Wenn wir uns das Alterssegment zwischen 20 und 39 Jahren anschauen (In der Schweiz sind das gem�ss BFS \footnote{\url{http://www.bfs.admin.ch/bfs/portal/de/index/themen/01/02/blank/key/alter/gesamt.html}} ca. 25\% der Bev�lkerung) bleiben ca. 24 Millionen Einwohner �brig.
Wir denken das Smoilet 2.0 haupts�chlich von Techniker oder Personen mit gleichrangigen berufen benutzt wird. Gem�ss BFS \footnote{\url{http://www.bfs.admin.ch/bfs/portal/de/index/themen/01/07/blank/ind43.indicator.43052.430108.html}} sind das ca 20\% und somit ca 5. Millionen Einwohner.

Das ergibt mit den oben genannten Zahlen (min. eine Toilette pro zwei Personen) etwa 2.5 Millionen Toiletten.

\section{Marktanteil}

F�r unsere Crowdfunding Kampagne streben wir den Verkauf von 10'000 Smoilet 2.0 Ger�ten an (0.4\% Marktanteil).
Anschliessend planen wir einen Wachstum von 50\% pro Jahr.

Das ergibt folgende Marktanteile.

1. Jahr 10'000 Einheiten 0.4 Prozent
2. Jahr 15'000 Einheiten 1 Prozent
3. Jahr 22'500 Einheiten 1.9 Prozent

% Todo Make Table!

\section{Bed�rfnisse}

Haushaltsger�te sollen heutzutage nicht mehr nur langweilige Ger�te sein, welche eine einfache Aufgabe erf�llen. K�hlschr�nke sind beispielsweise nicht mehr nur K�hlger�te. Sie denken mit und unterst�tzen den Besitzer bei den n�tigen Eink�ufen oder nehmen ihm sogar solche Aufgaben ab. So soll es auch mit der Toilette sein. Es gibt schon zahlreiche Produkte welche den Comfort der Toilettenbesucher erh�hen. Dies ist allerdings nicht das Ziel des Produktes. Smoilet soll Informationen zur Benutzung der Toilette liefern. Die Bed�rfnisse unserer Abnehmer lassen sich in folgende Kategorien unterteilen:
\begin{itemize}
\item �kologisch: Interesse an den Auswirkungen der Toilette auf die Umwelt.
\item Wirtschaftlich: M�gliches Sparpotenzial.
\item Wartung: Materialverbrauch, Abn�tzung.
\item Just for fun: Basteleien.
\end{itemize}

Diese Bed�rfnisse k�nnen alle komplett oder zum Teil durch unser Produkt abgedeckt werden.

\section{Konkurrenz}
Es gibt bereits zahlreiche Gadgets f�r die eigene Toilette zu kaufen. Die oben Beschriebenen Bed�rfnisse werden aber nicht oder kaum abgedeckt. Einzelne Prototypen wurden zwar schon entwickelt, sind aber nicht f�r den einfachen Toilettenbesitzer zug�nglich (https://twitter.com/shithappen, https://twitter.com/miketoilet, https://twitter.com/hacklabTOilet). 