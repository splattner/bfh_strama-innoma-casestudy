\chapter{XYZ}
\label{chap:XYZ}
2. Produkte, Dienstleistung

2.1. Marktleistung
Das Unternehmen startet mit einem Artikel im Angebot: Smoilet 2.0.
Smoilet 2.0 befriedigt die Bed�rfnisse von Toilettenenthusiasten aus aller Welt und wird deshalb �ber das Internet vertrieben. Auf Beratungen des Kunden vor Ort wird momentan verzichtet. Es wird auf eine schnell wachsende Comunity gesetzt, welche sich gegenseitig unterst�tzt und sich online auf unseren Foren �ber alle m�glichen Anwendungen des Produktes austauschen kann.

2.3. Abnehmer

Das Unternehmen ist vollst�ndig abh�ngig vom Endkonsumenten. Zu Beginn besteht ein grosses Potenzial vor allem bei Bastlern und anderweitig an Spielereien Interessierten Einzelpersonen. Wichtig f�r solche Personen ist ein m�glichst tiefer Preis, damit m�glichst allen der Zugang zu Smoilet 2.0 m�glich ist. Jeder Haushalt ist ein m�glicher Abnehmer des Produktes!
In einem weiteren Schritt sollen auch vermehrt Verwalter von �ffentlichen Toiletten als Abnehmer des Produktes ins Ziel genommen werden.

X. Bed�rfnisse
Haushaltsger�te sollen heutzutage nicht mehr nur langweilige Ger�te sein, welche eine einfache Aufgabe erf�llen. K�hlschr�nke sind beispielsweise nicht mehr nur K�hlger�te. Sie denken mit und unterst�tzen den Besitzer bei den n�tigen Eink�ufen oder nehmen ihm sogar solche Aufgaben ab. So soll es auch mit der Toilette sein. Es gibt schon zahlreiche Produkte welche den Comfort der Toilettenbesucher erh�hen. Dies ist allerdings nicht das Ziel des Produktes. Smoilet soll Informationen zur Benutzung der Toilette liefern. Die Bed�rfnisse unserer Abnehmer lassen sich in folgende Kategorien unterteilen:
- �kologisch: Interesse an den Auswirkungen der Toilette auf die Umwelt.
- Wirtschaftlich: M�gliches Sparpotenzial.
- Wartung: Materialverbrauch, Abn�tzung

Diese Bed�rfnisse k�nnen alle komplett oder zum Teil durch unser Produkt abgedeckt werden.

Konkurrenz
Es gibt bereits zahlreiche Gadgets f�r die eigene Toilette zu kaufen. Die oben Beschriebenen Bed�rfnisse werden aber nicht oder kaum abgedeckt. Einzelne Prototypen wurden zwar schon entwickelt, sind aber nicht f�r den einfachen Toilettenbesitzer zug�nglich (https://twitter.com/shithappen, https://twitter.com/miketoilet, https://twitter.com/hacklabTOilet). 