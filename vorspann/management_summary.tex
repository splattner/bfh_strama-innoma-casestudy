\chapter*{Management summary}
\label{chap:managementSummary}

Internet of Things (IOT) - ein neues Buzzword das aktuell nicht nur in der IT Welt Einzug findet, sondern auch immer wie mehr alltagstauglich wird. Insbesondere mit der Ank�ndigung von Swisscom ein schweizweites Netz f�r das Internet der Dinge aufzubauen, entdecken auch die weniger technikinteressierten Mitmenschen die Welt der \glqq Totalen-IT-Vernetzung\grqq.
Internet of Thing erlaubt es alle nur erdenklichen vorhandenen Gegenst�nde und Ger�te die bisher \glqq langweilig\grqq und analog waren, neues Leben einzuhauchen und intelligent zu machen.
Was w�re also naheliegender, als eines der wohl meist genutzten Objekte - die Toilette - Internet of Things f�hig zu machen. Damit ist unsere Idee \glqq Smoilet - Smart Toilet 2.0\grqq geboren. Smoilet - Smart Toilet 2.0 ist ein IOT Gadget das es erm�glicht, auf die Bedienung einer Toilettesp�lung smart zu reagieren. Damit �ffnet sich eine schier endlose Welt an M�glichkeiten. Von einer einfachen Aktivierung einer Toilettenl�ftung nach dem Sp�len, �ber Wasserverbrauchsanalyse bis hin zur automatisierten Verbrauchsmaterial Bestellung f�r �ffentliche Toilettenanlagen l�sst sich alles mit Smoilet - Smart Toilet 2.0 realisieren. Dieser Businessplan analysiert die wirtschaftliche Machbarkeit von Smoilet - Smart Toilet 2.0 durch Diskussion m�glicher Produktvarianten, Bewertung des Marktes und dessen m�glichen Marktanteils bzw. Absatzvolumen, der Ausarbeitung einer Preis und Absatzstrategie sowie durch die Erstellung einer soliden Finanzplanung.
